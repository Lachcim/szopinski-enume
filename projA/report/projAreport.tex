\documentclass{article}

\usepackage{amsmath}
\usepackage{multicol}
\setlength{\parskip}{1em}

\begin{document}

	\title{ENUME project report\\Project A: linear equation systems\\and matrix
	eigenvalues}
	\author{Michał Szopiński\\\\https://github.com/Lachcim/szopinski-enume}
	\date{November 12, 2020}
	\maketitle
	
	\numberwithin{equation}{section}
	
	\setcounter{section}{-1}
	\section{Abstract}
	
	This project explores the numerical methods of solving systems of linear
	equations and finding eigenvalues of matrices. The following concepts are
	discussed within this document:
	
	\begin{itemize}
		\item Computing the machine epsilon of an environment
		\item Solving linear equation systems using Gaussian elimination with
		partial pivoting
		\item Solving linear equation systems using the Jacobi and Gauss-Seidel
		algorithms
	\end{itemize}
	
	\newpage
	
	\section{Task 1: Computing the machine epsilon}
	
	\subsection{Overview}
	
	The goal of this task was to find the machine epsilon of the MATLAB
	environment, i.e. the maximum relative error of a floating-point
	representation of a number.
	
	\subsection{Implementation}
	
	The computation of the machine epsilon is based on the following two
	observations:
	
	\begin{enumerate}
		\item In binary computer environments, the epsilon is a negative power
		of two.
		\item The machine epsilon can also be expressed as:
		\begin{equation}
			eps = \min\{x \in M: fl(1 + x) > 1 \land x > 0\}
		\end{equation}
		i.e. the smallest representable number such that, when added to $1$,
		has a floating point representation greater than $1$.
	\end{enumerate}
	
	An arbitrary starting point, in this case $2^0$, can thus be chosen, and its
	value can then be sequentially halved until the criterion $fl(1 + x) > 1$ is
	no longer met.
	
	\subsection{Program output}
	
	\begin{verbatim}
		Found epsilon is 2.2204e-16
		Epsilon is 2.2204e-16
	\end{verbatim}
	
	\subsection{Observations}
	
	An epsilon of $2.2204 \cdot 10^{-16} = 2^{-52}$ is indicative of a standard
	double-precision IEEE 754 floating point binary representation.
	
	\newpage
	
	\section{Task 2: Solving systems of linear equations using Gaussian
	elimination}
	
	\subsection{Overview}
	
	The task consisted of two sub-tasks, each specifying a different system
	of equations to solve. The systems were to be solved using Gaussian
	elimination with partial pivoting. The size of the system was sequentially
	doubled until the computation time became prohibitive. The error of each
	solution was to be noted.
	
	Additionally, for systems whose number of equations was equal to 10,
	residual correction was to be applied and the solution was to be noted.
	
	The systems were given by:
	
	\noindent
	\begin{align*}
		&A_{ij}^{(a)} =
		\begin{cases} 
			4 & \text{for } i = j \\
			1 & \text{for } i = j \pm 1 \\
			0 & \text{otherwise}
		\end{cases}
		&&
		b_i^{(a)} = 4 + 0.3i
		\\
		&A_{ij}^{(b)} = \frac{6}{7(i + j + 1)}
		&&
		b_i^{(b)} =
		\begin{cases} 
			\frac{1}{3i} & \text{for even } i\\
			0 & \text{for odd } i
		\end{cases}
	\end{align*}
	
\end{document}
