\documentclass{article}

\usepackage{amsmath}
\usepackage{graphicx}
\usepackage{multicol}
\setlength{\parskip}{1em}

\begin{document}

	\title{ENUME project report\\Project B: finding roots of polynomials \\
	and zeros of other functions}
	\author{Michał Szopiński\\\\
	https://github.com/Lachcim/szopinski-enume\\
	Project number 60}
	\date{December 3, 2020}
	\maketitle
	
	\numberwithin{equation}{section}
	
	\setcounter{section}{-1}
	\section{Abstract}
	
	This project explores the numerical methods of finding the arguments (real
	and complex) for which a function is equal to zero. The following concepts
	are discussed within this document:
	
	\begin{itemize}
		\item Computing the zeros of an arbitrary function using the bisection
		method and Newton's algorithm
		\item Finding the real and complex roots of a
		4\textsuperscript{th}-degree polynomial using M{\"u}ller's method and
		Laguerre's algorithm
	\end{itemize}
	
	\newpage
	
	\section{Task 1: Finding the zeros of an arbitrary function}
	
	\subsection{Overview}
	
	In this task, the zeros of the following function were to be found:
	\begin{align*}
		f(x) = 0.7 x \cos(x) - \ln(x + 1) && x \in [2; 11]
	\end{align*}
	This was to be done using the bisection method and Newton's algorithm.
	
	\subsection{Implementation}
	
	\subsubsection{Bracketing}
	
	The basis for finding the roots of a function within a given interval is a
	heuristic algorithm which finds the function's \textit {root brackets},
	i.e. the regions where root-finding algorithms converge to an answer.
	
	In this project, bracketing was implemented by dividing the original search
	interval into 10 sub-intervals and checking the sign of the function at
	each of the sub-invervals' boundaries. A sign mismatch signalizes that the
	function crosses the x-axis somewhere within the interval, which can then
	be passed to a root-finding algorithm.
	
	\subsubsection{Bisection}
	
	Bisection is the simplest of root-finding algorithms. It can be thought of
	as a variation of the bracketing method. The original root bracket is
	divided into two sub-intervals and the next interval to be considered is
	chosen based on sign mismatch, implementing a ``binary search" type of
	approach. This step is repeated until the considered interval is
	sufficiently narrow.
	
	\subsubsection{Newton's method}
	
	A much more advanced algorithm is known as Newton's method. This method
	uses a linear approximation of the function at the given inquiry point to
	predict where the function may be zero. The next inquiry point is given as
	the intersection of the current tangent line and the x-axis.
	
	Because Newton's algorithm isn't strictly interval-based, a speed-based
	stop test has to be implemented. The test analyzes the distance between the
	current and the previous inquiry point and stops the algorithm when it is
	sufficiently small.
	
	\subsection{Program output}
	
	\subsubsection{Bisection method}
	
	\begin{multicols}{2}
		\begin{verbatim}
			step     root     value at root
			  1       5.15        -0.28874 
			  2      5.375         0.46222 
			  3     5.2625        0.091218 
			  4     5.2063       -0.098017 
			  5     5.2344       -0.003167 
			  6     5.2484         0.04409 
			  7     5.2414        0.020477 
			  8     5.2379       0.0086585 
			  9     5.2361       0.0027467 
			 10     5.2353     -0.00020995 
			 11     5.2357       0.0012684 
			 12     5.2355      0.00052925 
			 13     5.2354      0.00015965 
			 14     5.2353     -2.5147e-05 
			 15     5.2353      6.7253e-05 
			 16     5.2353      2.1053e-05 
			 17     5.2353     -2.0473e-06 
			 18     5.2353      9.5027e-06 
			 19     5.2353      3.7277e-06 
			 20     5.2353      8.4017e-07 
			 21     5.2353     -6.0359e-07 
			 22     5.2353      1.1829e-07 
			 23     5.2353     -2.4265e-07 
			 24     5.2353     -6.2178e-08 
			 25     5.2353      2.8056e-08 
			 26     5.2353     -1.7061e-08 
			 27     5.2353      5.4976e-09 
			 28     5.2353     -5.7818e-09 
			 29     5.2353      -1.421e-10 
			 30     5.2353      2.6777e-09 
			 31     5.2353      1.2678e-09 
			 32     5.2353      5.6286e-10 
			 33     5.2353      2.1038e-10 
			 34     5.2353      3.4142e-11 
			 35     5.2353      -5.398e-11 
			 36     5.2353     -9.9194e-12 
			 37     5.2353      1.2111e-11 
			 38     5.2353      1.0942e-12 
			 39     5.2353     -4.4122e-12 
			 40     5.2353     -1.6607e-12 
			 41     5.2353     -2.8288e-13 
			 42     5.2353      4.0723e-13 
			 43     5.2353      6.0618e-14 
			 44     5.2353      -1.128e-13 
			 45     5.2353     -2.5979e-14 
			 46     5.2353      1.8652e-14 
			 47     5.2353     -5.3291e-15 
			 48     5.2353      6.6613e-15 
			 49     5.2353      8.8818e-16 
			 50     5.2353     -2.2204e-15 
			 51     5.2353      8.8818e-16
		\end{verbatim}
	\end{multicols}
	
	\newpage
	
	\begin{multicols}{2}
		\begin{verbatim}
			step     root     value at root
			  1       7.85         -2.1585 
			  2      7.625        -0.94313 
			  3     7.5125        -0.38047 
			  4     7.4563         -0.1133 
			  5     7.4281        0.016422 
			  6     7.4422       -0.048105 
			  7     7.4352       -0.015758 
			  8     7.4316      0.00035306 
			  9     7.4334      -0.0076972 
			 10     7.4325      -0.0036708 
			 11     7.4321      -0.0016585 
			 12     7.4319     -0.00065265 
			 13     7.4318     -0.00014977 
			 14     7.4317      0.00010165 
			 15     7.4317     -2.4062e-05 
			 16     7.4317      3.8794e-05 
			 17     7.4317       7.366e-06 
			 18     7.4317     -8.3478e-06 
			 19     7.4317     -4.9087e-07 
			 20     7.4317      3.4376e-06 
			 21     7.4317      1.4734e-06 
			 22     7.4317      4.9124e-07 
			 23     7.4317      1.8718e-10 
			 24     7.4317     -2.4534e-07 
			 25     7.4317     -1.2258e-07 
			 26     7.4317     -6.1195e-08 
			 27     7.4317     -3.0504e-08 
			 28     7.4317     -1.5158e-08 
			 29     7.4317     -7.4856e-09 
			 30     7.4317     -3.6492e-09 
			 31     7.4317      -1.731e-09 
			 32     7.4317     -7.7191e-10 
			 33     7.4317     -2.9237e-10 
			 34     7.4317     -5.2595e-11 
			 35     7.4317      6.7295e-11 
			 36     7.4317      7.3497e-12 
			 37     7.4317     -2.2622e-11 
			 38     7.4317     -7.6343e-12 
			 39     7.4317     -1.4255e-13 
			 40     7.4317      3.6025e-12 
			 41     7.4317      1.7319e-12 
			 42     7.4317      7.9714e-13 
			 43     7.4317      3.2552e-13 
			 44     7.4317      8.9706e-14 
			 45     7.4317     -2.3981e-14 
			 46     7.4317      3.2419e-14 
			 47     7.4317      3.9968e-15 
			 48     7.4317     -7.9936e-15 
			 49     7.4317               0 
			 50     7.4317               0 
		\end{verbatim}
	\end{multicols}
	
	\subsubsection{Newton's method}
	
	\begin{multicols}{2}
		\begin{verbatim}
			step     root     value at root
			 1        5.15        -0.28874 
			 2      5.2349      -0.0012685 
			 3      5.2353     -4.1974e-08 
			 4      5.2353      8.8818e-16 
			 5      5.2353      8.8818e-16 
		\end{verbatim}
		\begin{verbatim}
			step     root     value at root
			 1        7.85         -2.1585 
			 2      7.4649        -0.15373 
			 3      7.4321      -0.0017786 
			 4      7.4317      -2.561e-07 
			 5      7.4317     -7.9936e-15 
			 6      7.4317               0 
			 7      7.4317               0 
		\end{verbatim}
	\end{multicols}
	
	\subsubsection{Function plots with marked zeros}
	
	\includegraphics[width=\textwidth]{bisectzeros}
	\includegraphics[width=\textwidth]{newtonzeros}
	
\end{document}
