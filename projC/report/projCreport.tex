\documentclass{article}

\usepackage{amsmath}
\usepackage{amssymb}
\usepackage{graphicx}
\usepackage{multicol}
\setlength{\parskip}{1em}

\begin{document}

	\title{ENUME project report\\Project C: least-squares approximation \\
	and solving systems of differential equations}
	\author{Michał Szopiński\\\\
	https://github.com/Lachcim/szopinski-enume\\
	Project number 60}
	\date{January 3, 2021}
	\maketitle
	
	\numberwithin{equation}{section}
	
	\setcounter{section}{-1}
	\section{Abstract}
	
	This project explores the numerical methods of function approximation
	and dynamic system analysis. The following concepts are discussed within
	this document:
	
	\begin{itemize}
		\item Performing the approximation of a function given by a set of
		discrete data points by a polynomial function of a given degree
		using the least-squares approach
		\item Determining the trajectory of an object whose motion is
		defined by a set of ordinary differential equations
	\end{itemize}
	
	\newpage
	
	\section{Task 1: Least-squares approximation}
	
	\subsection{Overview}
	
	The goal of this task was to find the approximation of a function given
	by a set of data points. The approximating function was to be a
	polynomial of varying degree (several degrees were to be tested). The
	least-squares problem was to be solved using QR factorization.
	
	\subsection{Implementation}
	
	\subsubsection{Finding the vector $\widehat{a}$}
	
	The approximating function is given by:
	\begin{equation}
		a_0 + a_1x + a_2x^2 + a_3x^3 + \ldots + a_nx^n = 0
	\end{equation}
	where $n$ is the degree of the approximating polynomial. The goal of the
	approximation is to find a vector $\widehat{a} = [ a_0; a_1; \ldots;
	a_n ]$ such that:
	\begin{equation}
		(\forall{a \in \mathbb{R}^n})\quad
		\left\Vert y - A\widehat{a} \right\Vert_2 \leq
		\left\Vert y - Aa \right\Vert_2
	\end{equation}
	where $A\widehat{a}$ forms the vector of values of the approximating
	function at each data point, and $y$ is the vector of the values of the
	data points. In other words, the goal is to minimize the Euclidean norm
	of the difference between the approximating function and the data
	points.
	
	The most optimal vector $\widehat{a}$ can be obtained by calculating the
	derivative of the error function with respect to $a_n$ and equating it
	to zero. The resulting system of equations, called the
	\textit{normal equations}, has a unique solution: the vector
	$\widehat{a}$.
	
	\subsubsection{Solving normal equations using QR decomposition}
	
	Although this system could be solved directly using Gaussian elimination
	or an iterative method, the matrix describing this system, called Gram's
	matrix, tends to have a high condition number, which limits the accuracy
	of the solution. For this reason, the matrix $A$ is defined:
	\begin{equation}
		A = \begin{bmatrix}
			1 & x_0 & (x_0)^2 & (x_0)^3 & \dots & (x_0)^n \\
			1 & x_1 & (x_1)^2 & (x_1)^3 & \dots & (x_1)^n \\
			1 & x_2 & (x_2)^2 & (x_2)^3 & \dots & (x_2)^n \\
			\vdots & \vdots & \vdots & \vdots & \ddots & \vdots \\
			1 & x_N & (x_N)^2 & (x_N)^3 & \dots & (x_N)^n
		\end{bmatrix}
	\end{equation}
	where $x_N$ is the x-axis position of the $N$-th data point. The system
	of natural equations can then be expressed in terms of $A$ as follows:
	\begin{equation}
		(A^TA)\widehat{a} = A^Ty
	\end{equation}
	The condition number of $A$ is equal to the square root of the condition
	number of the original Gram's matrix, which yields better numerical
	properties. Furthermore, this form lends itself nicely to being solved
	using QR factorization of $A$:
	\begin{equation}
		(R^TQ^T)(QR)\widehat{a} = (R^TQ^T)y
	\end{equation}
	By observing that $Q^TQ = I$ and $\det{R} \neq 0$, this equation can be
	further reduced to:
	\begin{equation}
		R\widehat{a} = Q^Ty
		\label{eq:rq}
	\end{equation}
	Because $R$ is an upper triangular matrix, the equation~\ref{eq:rq} can
	be easily solved using back-substitution.
	
\end{document}
